\newpage
{\Huge \bf Abstract}
\vspace{24pt} 

Intuitively towers of interpreters are a program architecture by which sequences of interpreters interpret each other and a user program is evaluated at the end of this chain. While one can imagine such
construct in everyday applications, prior research made use of towers of interpreters as a foundation to model reflection. As such, towers of interpreters in literature are synonymous with reflective towers and provide a tractable method with which to reason about reflection and design reflective languages. As a result, the assumptions and constraints that govern tower models make them unapplicable to practical or non-functional
settings. Prior formalizations of reflective towers have identified partial evaluation and reflection to harmonize in the development of such towers. We lift several restrictions of reflective towers including reflectivity, meta-circularity and homogeneity of data representation and then construct non-reflective towers
of interpreters to explore how partial evaluation techniques can be used to effectively remove levels of interpretation within such systems. %We then extend formalisms to such setting and go on to generalize previous techniques on partially evaluating towers of interpreters.

\newpage
\vspace*{\fill}
